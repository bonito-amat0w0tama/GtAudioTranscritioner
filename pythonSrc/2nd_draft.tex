\documentclass[twocolumn, a4paper]{ieicejsp}
\usepackage{newenum}
\usepackage{epsfig}
\usepackage{wrapfig}

\title{
    {\bf NMFを用いたギター演奏の自動採譜の検討\\}
    {\normalsize AN CONSIDERATION OF AUTOMATIC TRANSCRIPTION OF GUITAR PLAY USING NMF}
}

\author{
    大塚匡紀\\ Masaki Otsuka\and
    北原鉄朗\\ Tetsuro Kitahara
}

\affliate{
    日本大学大学院 総合基礎科学研究科 地球情報数理科学専攻\\
    Depatment of Computer Science and System Analysis, College of Humanities and Sciences, Nihon University
}
\renewcommand{\deg}{\mbox{$^\circ$}}

\begin{document}
\maketitle

\section{はじめに}
  ギター型のMIDI入力機器(MIDIギター)は、使い慣れた楽器で演奏情報を入力できる点で、DAWなどで作曲するギタリストには有用であるが、弦の振動をピックアップで取得するため、ピッキングの取りこぼしなどが発生し、MIDIキーボードに比べると入力される演奏情報の正確さに難がある。本研究では、非負値行列因子分解(NMF)による音響信号処理とMIDIギターピックアップによる処理とを統合することで、MIDIデータ化(本稿ではこれを「採譜」と称する)の高精度化を目指す。本稿では、その予備検討として、同一旋律に対するNMFによる採譜結果とMIDIギターピックアップによる採譜結果とを比較し、得意・不得意な点を考察する。

\section{NMFによる採譜手法}
\subsection{予備演奏による基底ベクトルの推定}
本演奏に先立ち、本演奏に用いるのと全く同じギターを用いて、各弦の各フレットを一音ずつ順番に演奏する。その演奏のスペクトログラムVを求める(サンプリング周波数:44100Hz、窓幅:8092、シフト幅:xxxx)と、NMFを用いてVを2つの非負値行列\texbf{W}, \texbf{H}の積に分解する:
\begin{equation}
    \mathbf{V} \cong \mathbf{W}\mathbf{H}
\end{equation}
Wはxxxxxx、Hはxxxxxxを表す。
(各基底に対してMIDIノートナンバーを関連付ける方法を述べる。場合によっては基底の統合も。)



本研究では、音響信号解析での自動採譜にNMFを用いた。以下より本研究で用いた数式を示す。短時間フーリエ変換(STFT)によって得られたスペクトログラムを非負値行列\textbf{V}とする。そして、\textbf{V}をNMFを用いて分解すると、非負値行列\textbf{W}と非負値行列\textbf{H}の積の形となり、以下のようになる。
\begin{equation}
    \mathbf{V} \cong \mathbf{W}\mathbf{H}
\end{equation}
しかし、NMFはリアルタイムでの自動採譜には対応していない。そこで本研究では、採譜前にギターの全フレットの音(異弦同音区別する)を1音づつ演奏することを条件とし、以下のような数式に変形することによりjリアルタイムでの自動採譜に対応した。
\begin{equation}
    \mathbf{W}^{-1}\mathbf{V} \cong \mathbf{H}
\end{equation}
\begin{equation}
    \mathbf{W}^{-1}\mathbf{v} \cong \mathbf{h}
\end{equation}
(2)の式における\textbf{v}はリアルタイムに入力されるギター演奏を表しており、\textbf{h}は入力されたギター演奏のリズムを表している。ここで、\textbf{W}は正方行列ではないため、\textbf{W}-1は擬似逆行列となる。

\section{実験結果}
\subsection{方法}

\section{おわりに}


\begin{thebibliography}{9} \itemsep=-1mm
\bibitem{Oh2012} 黄 楊暘 他: ``マイクロホンアレイを用いた複数人対話からの発話区間検出 および話者方向推定の評価手法'', 
ロボット学会学術講演会, 3D1-4, 2012.

\bibitem{hark}http://winnie.kuis.kyoto-u.ac.jp/HARK/

\bibitem{Schmidt1986} R. O. Schmidt: ``Multiple Emitter Location and Signal Parameter Estimation'', {\it IEEE Trans. Antennas and Propagation}, {\bf 34}, 3, pp.276--280, 1986.
\end{thebibliography}

\end{document}
